\documentclass[]{article}
\usepackage{lmodern}
\usepackage{amssymb,amsmath}
\usepackage{ifxetex,ifluatex}
\usepackage{fixltx2e} % provides \textsubscript
\ifnum 0\ifxetex 1\fi\ifluatex 1\fi=0 % if pdftex
  \usepackage[T1]{fontenc}
  \usepackage[utf8]{inputenc}
\else % if luatex or xelatex
  \ifxetex
    \usepackage{mathspec}
  \else
    \usepackage{fontspec}
  \fi
  \defaultfontfeatures{Ligatures=TeX,Scale=MatchLowercase}
    \setmainfont[]{NanumGothic}
\fi
% use upquote if available, for straight quotes in verbatim environments
\IfFileExists{upquote.sty}{\usepackage{upquote}}{}
% use microtype if available
\IfFileExists{microtype.sty}{%
\usepackage{microtype}
\UseMicrotypeSet[protrusion]{basicmath} % disable protrusion for tt fonts
}{}
\usepackage[margin=1in]{geometry}
\usepackage{hyperref}
\hypersetup{unicode=true,
            pdftitle={homework3},
            pdfauthor={Hyeonho Lee},
            pdfborder={0 0 0},
            breaklinks=true}
\urlstyle{same}  % don't use monospace font for urls
\usepackage{color}
\usepackage{fancyvrb}
\newcommand{\VerbBar}{|}
\newcommand{\VERB}{\Verb[commandchars=\\\{\}]}
\DefineVerbatimEnvironment{Highlighting}{Verbatim}{commandchars=\\\{\}}
% Add ',fontsize=\small' for more characters per line
\usepackage{framed}
\definecolor{shadecolor}{RGB}{248,248,248}
\newenvironment{Shaded}{\begin{snugshade}}{\end{snugshade}}
\newcommand{\KeywordTok}[1]{\textcolor[rgb]{0.13,0.29,0.53}{\textbf{#1}}}
\newcommand{\DataTypeTok}[1]{\textcolor[rgb]{0.13,0.29,0.53}{#1}}
\newcommand{\DecValTok}[1]{\textcolor[rgb]{0.00,0.00,0.81}{#1}}
\newcommand{\BaseNTok}[1]{\textcolor[rgb]{0.00,0.00,0.81}{#1}}
\newcommand{\FloatTok}[1]{\textcolor[rgb]{0.00,0.00,0.81}{#1}}
\newcommand{\ConstantTok}[1]{\textcolor[rgb]{0.00,0.00,0.00}{#1}}
\newcommand{\CharTok}[1]{\textcolor[rgb]{0.31,0.60,0.02}{#1}}
\newcommand{\SpecialCharTok}[1]{\textcolor[rgb]{0.00,0.00,0.00}{#1}}
\newcommand{\StringTok}[1]{\textcolor[rgb]{0.31,0.60,0.02}{#1}}
\newcommand{\VerbatimStringTok}[1]{\textcolor[rgb]{0.31,0.60,0.02}{#1}}
\newcommand{\SpecialStringTok}[1]{\textcolor[rgb]{0.31,0.60,0.02}{#1}}
\newcommand{\ImportTok}[1]{#1}
\newcommand{\CommentTok}[1]{\textcolor[rgb]{0.56,0.35,0.01}{\textit{#1}}}
\newcommand{\DocumentationTok}[1]{\textcolor[rgb]{0.56,0.35,0.01}{\textbf{\textit{#1}}}}
\newcommand{\AnnotationTok}[1]{\textcolor[rgb]{0.56,0.35,0.01}{\textbf{\textit{#1}}}}
\newcommand{\CommentVarTok}[1]{\textcolor[rgb]{0.56,0.35,0.01}{\textbf{\textit{#1}}}}
\newcommand{\OtherTok}[1]{\textcolor[rgb]{0.56,0.35,0.01}{#1}}
\newcommand{\FunctionTok}[1]{\textcolor[rgb]{0.00,0.00,0.00}{#1}}
\newcommand{\VariableTok}[1]{\textcolor[rgb]{0.00,0.00,0.00}{#1}}
\newcommand{\ControlFlowTok}[1]{\textcolor[rgb]{0.13,0.29,0.53}{\textbf{#1}}}
\newcommand{\OperatorTok}[1]{\textcolor[rgb]{0.81,0.36,0.00}{\textbf{#1}}}
\newcommand{\BuiltInTok}[1]{#1}
\newcommand{\ExtensionTok}[1]{#1}
\newcommand{\PreprocessorTok}[1]{\textcolor[rgb]{0.56,0.35,0.01}{\textit{#1}}}
\newcommand{\AttributeTok}[1]{\textcolor[rgb]{0.77,0.63,0.00}{#1}}
\newcommand{\RegionMarkerTok}[1]{#1}
\newcommand{\InformationTok}[1]{\textcolor[rgb]{0.56,0.35,0.01}{\textbf{\textit{#1}}}}
\newcommand{\WarningTok}[1]{\textcolor[rgb]{0.56,0.35,0.01}{\textbf{\textit{#1}}}}
\newcommand{\AlertTok}[1]{\textcolor[rgb]{0.94,0.16,0.16}{#1}}
\newcommand{\ErrorTok}[1]{\textcolor[rgb]{0.64,0.00,0.00}{\textbf{#1}}}
\newcommand{\NormalTok}[1]{#1}
\usepackage{longtable,booktabs}
\usepackage{graphicx,grffile}
\makeatletter
\def\maxwidth{\ifdim\Gin@nat@width>\linewidth\linewidth\else\Gin@nat@width\fi}
\def\maxheight{\ifdim\Gin@nat@height>\textheight\textheight\else\Gin@nat@height\fi}
\makeatother
% Scale images if necessary, so that they will not overflow the page
% margins by default, and it is still possible to overwrite the defaults
% using explicit options in \includegraphics[width, height, ...]{}
\setkeys{Gin}{width=\maxwidth,height=\maxheight,keepaspectratio}
\IfFileExists{parskip.sty}{%
\usepackage{parskip}
}{% else
\setlength{\parindent}{0pt}
\setlength{\parskip}{6pt plus 2pt minus 1pt}
}
\setlength{\emergencystretch}{3em}  % prevent overfull lines
\providecommand{\tightlist}{%
  \setlength{\itemsep}{0pt}\setlength{\parskip}{0pt}}
\setcounter{secnumdepth}{0}
% Redefines (sub)paragraphs to behave more like sections
\ifx\paragraph\undefined\else
\let\oldparagraph\paragraph
\renewcommand{\paragraph}[1]{\oldparagraph{#1}\mbox{}}
\fi
\ifx\subparagraph\undefined\else
\let\oldsubparagraph\subparagraph
\renewcommand{\subparagraph}[1]{\oldsubparagraph{#1}\mbox{}}
\fi

%%% Use protect on footnotes to avoid problems with footnotes in titles
\let\rmarkdownfootnote\footnote%
\def\footnote{\protect\rmarkdownfootnote}

%%% Change title format to be more compact
\usepackage{titling}

% Create subtitle command for use in maketitle
\newcommand{\subtitle}[1]{
  \posttitle{
    \begin{center}\large#1\end{center}
    }
}

\setlength{\droptitle}{-2em}

  \title{homework3}
    \pretitle{\vspace{\droptitle}\centering\huge}
  \posttitle{\par}
    \author{Hyeonho Lee}
    \preauthor{\centering\large\emph}
  \postauthor{\par}
      \predate{\centering\large\emph}
  \postdate{\par}
    \date{2018년 8월 23일}

\usepackage{kotex}

\begin{document}
\maketitle

\section{~}\label{section}

Exercises 8\\
Repeat exercise \#3. This time, use the variables cohort, zygosity to
facet, and use different colors to indicate the subgroups for which the
heights are significantly similar. Comment on your finding.

~

\begin{Shaded}
\begin{Highlighting}[]
\NormalTok{condition =}\StringTok{ }\NormalTok{twinData }\OperatorTok\StringTok{ }\KeywordTok{group_by}\NormalTok{(cohort,zygosity) }\OperatorTok\StringTok{ }
\StringTok{  }\KeywordTok{do}\NormalTok{(}\KeywordTok{tidy}\NormalTok{( }\KeywordTok{cor.test}\NormalTok{(}\OperatorTok{~}\StringTok{ }\NormalTok{ht1 }\OperatorTok{+}\StringTok{ }\NormalTok{ht2, }\DataTypeTok{alternative =} \StringTok{"greater"}\NormalTok{ , }\DataTypeTok{data =}\NormalTok{ . ))) }\OperatorTok\StringTok{ }
\StringTok{  }\KeywordTok{select}\NormalTok{(cohort,zygosity,estimate)}

\NormalTok{twinData =}\StringTok{ }\KeywordTok{merge}\NormalTok{(twinData, condition, }\DataTypeTok{by =} \KeywordTok{c}\NormalTok{(}\StringTok{'cohort'}\NormalTok{, }\StringTok{'zygosity'}\NormalTok{))}

\NormalTok{twinData }\OperatorTok\StringTok{ }\KeywordTok{ggplot}\NormalTok{(}\DataTypeTok{mapping =} \KeywordTok{aes}\NormalTok{(ht1,ht2,}\DataTypeTok{color=}\NormalTok{estimate)) }\OperatorTok{+}\StringTok{ }
\StringTok{  }\KeywordTok{geom_point}\NormalTok{() }\OperatorTok{+}\StringTok{ }\KeywordTok{facet_grid}\NormalTok{(}\DataTypeTok{rows =} \KeywordTok{vars}\NormalTok{(cohort), }\DataTypeTok{cols =} \KeywordTok{vars}\NormalTok{(zygosity))}
\end{Highlighting}
\end{Shaded}

\includegraphics{Hyeonho_Lee-HW3_files/figure-latex/unnamed-chunk-2-1.pdf}
상관계수가 0.5이상은 그룹은 MZFF, MZMM, older, younger, DZMM의 older
5개의 그룹이 있다. 나머지 그룹은 0.5미만임을 알 수 있다.

\section{}\label{section-1}

Exercises 9\\
Repeat exercise \#8, but compare weight this time. You should be able to
recycle almost all code chunks.

~

\begin{Shaded}
\begin{Highlighting}[]
\NormalTok{condition2 =}\StringTok{ }\NormalTok{twinData }\OperatorTok\StringTok{ }\KeywordTok{group_by}\NormalTok{(cohort,zygosity) }\OperatorTok\StringTok{  }
\StringTok{  }\KeywordTok{do}\NormalTok{(}\KeywordTok{tidy}\NormalTok{( }\KeywordTok{cor.test}\NormalTok{(}\OperatorTok{~}\StringTok{ }\NormalTok{wt1 }\OperatorTok{+}\StringTok{ }\NormalTok{wt2, }\DataTypeTok{alternative =} \StringTok{"greater"}\NormalTok{ , }\DataTypeTok{data =}\NormalTok{ . ))) }\OperatorTok\StringTok{ }
\StringTok{  }\KeywordTok{select}\NormalTok{(cohort,zygosity,estimate)}

\NormalTok{twinData =}\StringTok{ }\NormalTok{twinData }\OperatorTok\StringTok{ }\KeywordTok{select}\NormalTok{(}\OperatorTok{-}\NormalTok{estimate) }\OperatorTok\StringTok{ }\KeywordTok{merge}\NormalTok{(condition2, }\DataTypeTok{by =} \KeywordTok{c}\NormalTok{(}\StringTok{'cohort'}\NormalTok{, }\StringTok{'zygosity'}\NormalTok{))}

\NormalTok{twinData }\OperatorTok\StringTok{ }\KeywordTok{ggplot}\NormalTok{(}\DataTypeTok{mapping =} \KeywordTok{aes}\NormalTok{(wt1,wt2,}\DataTypeTok{color=}\NormalTok{estimate)) }\OperatorTok{+}\StringTok{ }
\StringTok{  }\KeywordTok{geom_point}\NormalTok{() }\OperatorTok{+}\StringTok{ }\KeywordTok{facet_grid}\NormalTok{(}\DataTypeTok{rows =} \KeywordTok{vars}\NormalTok{(cohort), }\DataTypeTok{cols =} \KeywordTok{vars}\NormalTok{(zygosity))}
\end{Highlighting}
\end{Shaded}

\includegraphics{Hyeonho_Lee-HW3_files/figure-latex/unnamed-chunk-3-1.pdf}
4개의 그룹만이(MZFF, MZMM) 0.5이상의 상관관계가 있다. \# ~

\section{~}\label{section-2}

\subsection{~}\label{section-3}

~

~

Exercises 10\\
Recreate the following graphic. This involves transforming twinData into
a narrow form using gather(). You might want to take a look at Lecture 7
note for boxplots.

~

\begin{Shaded}
\begin{Highlighting}[]
\NormalTok{twinData_}\DecValTok{1}\NormalTok{ =}\StringTok{ }\NormalTok{twinData }\OperatorTok\StringTok{ }\KeywordTok{select}\NormalTok{(ht1, ht2) }\OperatorTok\StringTok{ }\KeywordTok{gather}\NormalTok{(}\DataTypeTok{key =} \StringTok{"order"}\NormalTok{, }\DataTypeTok{value =} \StringTok{"heigth"}\NormalTok{) }\OperatorTok\StringTok{ }
\StringTok{  }\KeywordTok{cbind}\NormalTok{(twinData}\OperatorTok{$}\NormalTok{cohort, twinData}\OperatorTok{$}\NormalTok{zygosity)}
\KeywordTok{colnames}\NormalTok{(twinData_}\DecValTok{1}\NormalTok{)[}\DecValTok{3}\OperatorTok{:}\DecValTok{4}\NormalTok{] =}\StringTok{ }\KeywordTok{c}\NormalTok{(}\StringTok{'cohort'}\NormalTok{, }\StringTok{'zygosity'}\NormalTok{)}

\NormalTok{twinData_}\DecValTok{1} \OperatorTok\StringTok{ }\KeywordTok{na.omit}\NormalTok{() }\OperatorTok\StringTok{ }\KeywordTok{ggplot}\NormalTok{(}\DataTypeTok{mapping =} \KeywordTok{aes}\NormalTok{(order, heigth)) }\OperatorTok{+}
\StringTok{  }\KeywordTok{geom_boxplot}\NormalTok{() }\OperatorTok{+}\StringTok{ }\KeywordTok{facet_grid}\NormalTok{(}\DataTypeTok{rows =} \KeywordTok{vars}\NormalTok{(cohort), }\DataTypeTok{cols =} \KeywordTok{vars}\NormalTok{(zygosity))}
\end{Highlighting}
\end{Shaded}

\includegraphics{Hyeonho_Lee-HW3_files/figure-latex/unnamed-chunk-4-1.pdf}

\section{~}\label{section-4}

Exercises 11\\
Inspect the data graphic. Is there any need to adjust the hypothesis
(posed in Question \#2)?\\
Are the first-born taller than the second-born?

전체 데이터를 보았을 때 상관성이 있어 보였으나, 각 범주로 나누어 비교해
본 결과, DZOS 경우에만 키의 차이를 보이는 것을 볼 수 있다.

~

~

~

~

Exercises 12

Use the paired t-test to test whether the first-born's height is
significantly different from the second-born for each subgroup.

Why do we use the paired t-test, as opposed to using the two-sample
t-test?

\begin{Shaded}
\begin{Highlighting}[]
\NormalTok{twinData_t.test =}\StringTok{ }\NormalTok{twinData }\OperatorTok\StringTok{ }\KeywordTok{group_by}\NormalTok{(cohort, zygosity) }\OperatorTok\StringTok{ }
\StringTok{  }\KeywordTok{do}\NormalTok{(}\KeywordTok{tidy}\NormalTok{(}\KeywordTok{t.test}\NormalTok{(.}\OperatorTok{$}\NormalTok{ht1, .}\OperatorTok{$}\NormalTok{ht2, }\DataTypeTok{data=}\NormalTok{., }\DataTypeTok{paired =} \OtherTok{TRUE}\NormalTok{)))}

\NormalTok{knitr}\OperatorTok{::}\KeywordTok{kable}\NormalTok{(twinData_t.test[,}\KeywordTok{c}\NormalTok{(}\DecValTok{1}\OperatorTok{:}\DecValTok{5}\NormalTok{,}\DecValTok{9}\NormalTok{,}\DecValTok{10}\NormalTok{)], }\DataTypeTok{caption =} \StringTok{'t-test'}\NormalTok{)}
\end{Highlighting}
\end{Shaded}

\begin{longtable}[]{@{}llrrrll@{}}
\caption{t-test}\tabularnewline
\toprule
cohort & zygosity & estimate & statistic & p.value & method &
alternative\tabularnewline
\midrule
\endfirsthead
\toprule
cohort & zygosity & estimate & statistic & p.value & method &
alternative\tabularnewline
\midrule
\endhead
older & MZFF & 0.0012716 & 0.9528187 & 0.3410400 & Paired t-test &
two.sided\tabularnewline
older & MZMM & -0.0002389 & -0.1314739 & 0.8954925 & Paired t-test &
two.sided\tabularnewline
older & DZFF & 0.0031411 & 0.9159889 & 0.3602417 & Paired t-test &
two.sided\tabularnewline
older & DZMM & -0.0063464 & -1.2020046 & 0.2314056 & Paired t-test &
two.sided\tabularnewline
older & DZOS & -0.1411766 & -40.4927683 & 0.0000000 & Paired t-test &
two.sided\tabularnewline
younger & MZFF & 0.0001800 & 0.1280849 & 0.8981288 & Paired t-test &
two.sided\tabularnewline
younger & MZMM & 0.0012818 & 0.6346907 & 0.5261947 & Paired t-test &
two.sided\tabularnewline
younger & DZFF & 0.0076053 & 1.9417935 & 0.0529878 & Paired t-test &
two.sided\tabularnewline
younger & DZMM & 0.0021255 & 0.3758702 & 0.7074305 & Paired t-test &
two.sided\tabularnewline
younger & DZOS & -0.1429043 & -43.1645170 & 0.0000000 & Paired t-test &
two.sided\tabularnewline
\bottomrule
\end{longtable}

쌍둥이 이기 때문에 한 집단에 대한 비교로 생각 할 수 있고, 그러므로
paired 비교가 적합하다.

~

~

Test whether there is a difference with significance level 5\%, for each
subgroup.

\begin{Shaded}
\begin{Highlighting}[]
\NormalTok{twinData_t.test =}\StringTok{ }\NormalTok{twinData }\OperatorTok\StringTok{ }\KeywordTok{group_by}\NormalTok{(cohort, zygosity) }\OperatorTok\StringTok{ }
\StringTok{  }\KeywordTok{do}\NormalTok{(}\KeywordTok{tidy}\NormalTok{(}\KeywordTok{t.test}\NormalTok{(.}\OperatorTok{$}\NormalTok{ht1, .}\OperatorTok{$}\NormalTok{ht2, }\DataTypeTok{data=}\NormalTok{., }\DataTypeTok{paired =} \OtherTok{TRUE}\NormalTok{, }\DataTypeTok{conf.level =} \FloatTok{0.95}\NormalTok{)))}

\NormalTok{knitr}\OperatorTok{::}\KeywordTok{kable}\NormalTok{(twinData_t.test[,}\KeywordTok{c}\NormalTok{(}\DecValTok{1}\NormalTok{,}\DecValTok{2}\NormalTok{,}\DecValTok{3}\NormalTok{,}\DecValTok{7}\NormalTok{,}\DecValTok{8}\NormalTok{)], }\DataTypeTok{caption =} \StringTok{'significance level 5%'}\NormalTok{)}
\end{Highlighting}
\end{Shaded}

\begin{longtable}[]{@{}llrrr@{}}
\caption{significance level 5\%}\tabularnewline
\toprule
cohort & zygosity & estimate & conf.low & conf.high\tabularnewline
\midrule
\endfirsthead
\toprule
cohort & zygosity & estimate & conf.low & conf.high\tabularnewline
\midrule
\endhead
older & MZFF & 0.0012716 & -0.0013490 & 0.0038922\tabularnewline
older & MZMM & -0.0002389 & -0.0038152 & 0.0033375\tabularnewline
older & DZFF & 0.0031411 & -0.0036011 & 0.0098833\tabularnewline
older & DZMM & -0.0063464 & -0.0167857 & 0.0040928\tabularnewline
older & DZOS & -0.1411766 & -0.1480318 & -0.1343213\tabularnewline
younger & MZFF & 0.0001800 & -0.0025799 & 0.0029399\tabularnewline
younger & MZMM & 0.0012818 & -0.0026952 & 0.0052587\tabularnewline
younger & DZFF & 0.0076053 & -0.0000986 & 0.0153091\tabularnewline
younger & DZMM & 0.0021255 & -0.0090286 & 0.0132797\tabularnewline
younger & DZOS & -0.1429043 & -0.1494094 & -0.1363993\tabularnewline
\bottomrule
\end{longtable}

Finally, recreate the above graphic with different colors indicating the
results of t-tests.

\begin{Shaded}
\begin{Highlighting}[]
\NormalTok{twinData_}\DecValTok{1}\NormalTok{ =}\StringTok{ }\KeywordTok{merge}\NormalTok{(twinData_}\DecValTok{1}\NormalTok{, twinData_t.test, }\DataTypeTok{by =} \KeywordTok{c}\NormalTok{(}\StringTok{'cohort'}\NormalTok{, }\StringTok{'zygosity'}\NormalTok{))}

\NormalTok{twinData_}\DecValTok{1} \OperatorTok\StringTok{ }\KeywordTok{ggplot}\NormalTok{(}\DataTypeTok{mapping =} \KeywordTok{aes}\NormalTok{(order, heigth, }\DataTypeTok{fill =}\NormalTok{ p.value)) }\OperatorTok{+}
\StringTok{  }\KeywordTok{geom_boxplot}\NormalTok{() }\OperatorTok{+}\StringTok{ }\KeywordTok{facet_grid}\NormalTok{(}\DataTypeTok{rows =} \KeywordTok{vars}\NormalTok{(cohort), }\DataTypeTok{cols =} \KeywordTok{vars}\NormalTok{(zygosity))}
\end{Highlighting}
\end{Shaded}

\includegraphics{Hyeonho_Lee-HW3_files/figure-latex/unnamed-chunk-7-1.pdf}


\end{document}
